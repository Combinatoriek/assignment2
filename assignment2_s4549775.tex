\documentclass[12pt]{article}

\usepackage{amsmath}
\usepackage{amssymb}
\usepackage{enumerate}
\usepackage[margin=5em]{geometry}

\setlength\parindent{0em}
\setlength\parskip{1em}

\title{Assignment 2}

\author{
	Hendrik Werner s4549775
}

\begin{document}
\maketitle

\section*{10}
If the committee needs to contain more women than men (which is sexist btw), and it consists of 6 members, then it must contain 4, 5, or 6 women.

\begin{tabular}{r|l}
	women & possibilities\\\hline
	4 & $C(15, 4) * C(10, 2)$\\
	5 & $C(15, 5) * C(10, 1)$\\
	6 & $C(15, 6) * C(10, 0)$
\end{tabular}

Total number of ways to form the committee: $\sum_{i=4}^{6} C(15, i) * C(10, 6 - i) = $

\section*{11}

\section*{12}
Pascal's identity says: $=$ so we must add the numbers above together to get the numbers in the next row.

$0 + 1 = 1$, $1 + 8 = 9$, $8 + 28 = 36$, $28 + 56 = 84$, $56 + 70 = 126$, $70 + 56 = 126$, $56 + 28 = 84$, $28 + 8 = 36$, $8 + 1 = 9$, $1 + 0 = 1$

The next row is $1\ 9\ 36\ 84\ 126\ 126\ 84\ 36\ 9\ 1$.

\section*{13}
\begin{enumerate}[a]
	\item %a
	\item %b
\end{enumerate}

\section*{14}

\section*{15}
\begin{enumerate}[a]
	\item %a
	\item %b
	\item %c
\end{enumerate}

\end{document}
