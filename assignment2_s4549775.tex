\documentclass[12pt]{article}

\usepackage{amsmath}
\usepackage{amssymb}
\usepackage{enumerate}
\usepackage[margin=5em]{geometry}

\setlength\parindent{0em}
\setlength\parskip{1em}

\title{Assignment 2}

\author{
	Hendrik Werner s4549775
}

\begin{document}
\maketitle

\section*{10}
If the committee needs to contain more women than men (which is sexist btw), and it consists of 6 members, then it must contain 4, 5, or 6 women.

\begin{tabular}{r|l}
	women & possibilities\\\hline
	4 & $C(15, 4) * C(10, 2)$\\
	5 & $C(15, 5) * C(10, 1)$\\
	6 & $C(15, 6) * C(10, 0)$
\end{tabular}

Total number of ways to form the committee: $\sum_{i=4}^{6} C(15, i) * C(10, 6 - i) = $

\section*{11}

\section*{12}
Pascal's identity says: $=$ so we must add the numbers above together to get the numbers in the next row.

$0 + 1 = 1$, $1 + 8 = 9$, $8 + 28 = 36$, $28 + 56 = 84$, $56 + 70 = 126$, $70 + 56 = 126$, $56 + 28 = 84$, $28 + 8 = 36$, $8 + 1 = 9$, $1 + 0 = 1$

The next row is $1\ 9\ 36\ 84\ 126\ 126\ 84\ 36\ 9\ 1$.

\section*{13}
\begin{enumerate}[a]
	\item %a
	In the first case you choose $r$ elements out of a set of $n$ elements, next you choose $k$ elements from the set you chose in step one.\\
	In the end you are left with a pile of $k$ elements that were chosen in step 2, and a pile of $r - k$ elements, that were not chosen in step 2.

	The second method is a bit more direct. You immediately choose $k$ elements from a set of $n$ elements. This is the same as the first pile from method one. Now you are left with $n - k$ elements, of which you choose $r - k$ to match the pile that you chose in step 1 of method 1, but that wasn't chosen in step 2.\\
	In the end you are again left with two piles of identical sizes as in method 1.

	Both methods of choosing elements are equivalent.

	\item %b
	$C(n, r) * C(r, k)\\
	= \dfrac{n!}{r! (n - r)!} * \dfrac{r!}{k! (r - k)!}\\
	= \dfrac{n! r!}{r! (n - r)! k! (r - k)!}\\
	= \dfrac{n!}{(n - r)! k! (r - k)!}\\
	= \dfrac{n!}{k! (r - k)! (n - r)!}\\
	= \dfrac{n! (n - k)!}{k! (n - k)! (r - k)! (n - r)!}\\
	= \dfrac{n!}{k! (n - k)!} * \dfrac{(n - k)!}{(r - k)! (n - r)!}\\
	= \dfrac{n!}{k! (n - k)!} * \dfrac{(n - k)!}{(r - k)! (n - k - r + k)!}\\
	= C(n, k) * C(n - k, r - k)$
\end{enumerate}

\section*{14}

\section*{15}
\begin{enumerate}[a]
	\item %a
	\item %b
	\item %c
\end{enumerate}

\end{document}
