\documentclass[12pt]{article}

\usepackage{amsmath}
\usepackage{amssymb}
\usepackage{enumerate}
\usepackage[margin=5em]{geometry}

\setlength\parindent{0em}
\setlength\parskip{1em}

\title{
	Assignment 2\\
	Group 4
}

\author{
	Hendrik Werner s4549775
}

\begin{document}
\maketitle

\section*{10}
If the committee needs to contain more women than men (which is sexist btw), and it consists of 6 members, then it must contain 4, 5, or 6 women.

\begin{tabular}{r|l}
	women & possibilities\\\hline
	4 & $C(15, 4) * C(10, 2) = 61425$\\
	5 & $C(15, 5) * C(10, 1) = 30030$\\
	6 & $C(15, 6) * C(10, 0) = 5005$
\end{tabular}

Total number of ways to form the committee: $\sum_{i=4}^{6} C(15, i) * C(10, 6 - i) = 96460$

\section*{11}
There are lots of different possible scenarios. We can enumerate them and add the possibilities of each case. G indicates a gold medal, S a silver medal, B a bronze medal, and L no medal.

\begin{tabular}{r|l}
	Medals & Possibilities\\\hline
	\texttt{GGGGGG} & $C(6, 6) = 1$\\
	\texttt{GGGGGL} & $C(6, 5) = 6$\\
	\texttt{GGGGLL} & $C(6, 4) = 15$\\
	\texttt{GGGLLL} & $C(6, 3) = 20$\\
	\texttt{GGBBBB} & $C(6, 2) * C(4, 4) = 15$\\
	\texttt{GGBBBL} & $C(6, 2) * C(4, 3) = 60$\\
	\texttt{GGBBLL} & $C(6, 2) * C(4, 2) = 90$\\
	\texttt{GGBLLL} & $C(6, 2) * C(4, 1) = 60$\\
	\texttt{GSSSSS} & $C(6, 1) * C(5, 5) = 6$\\
	\texttt{GSSSSL} & $C(6, 1) * C(5, 4) = 30$\\
	\texttt{GSSSLL} & $C(6, 1) * C(5, 3) = 60$\\
	\texttt{GSSLLL} & $C(6, 1) * C(5, 2) = 60$\\
	\texttt{GSBBBB} & $C(6, 1) * C(5, 1) * C(4, 4) = 30$\\
	\texttt{GSBBBL} & $C(6, 1) * C(5, 1) * C(4, 3) = 120$\\
	\texttt{GSBBLL} & $C(6, 1) * C(5, 1) * C(4, 2) = 180$\\
	\texttt{GSBLLL} & $C(6, 1) * C(5, 1) * C(4, 1) = 120$\\
\end{tabular}

Total possibilities to award medals: $1 + 6 + 15 + 20 + 15 + 60 + 90 + 60 + 6 + 30 + 60 + 60 + 30 + 120 + 180 + 120 = 873$.

\section*{12}
Pascal's identity says: $\begin{pmatrix}
	n \\ k
\end{pmatrix}
= \begin{pmatrix}
	n - 1 \\ k - 1
\end{pmatrix}
+ \begin{pmatrix}
	n - 1 \\ k
\end{pmatrix}$ so we must add the numbers above together to get the numbers in the next row.

$0 + 1 = 1$, $1 + 8 = 9$, $8 + 28 = 36$, $28 + 56 = 84$, $56 + 70 = 126$, $70 + 56 = 126$, $56 + 28 = 84$, $28 + 8 = 36$, $8 + 1 = 9$, $1 + 0 = 1$

The next row is $1\ 9\ 36\ 84\ 126\ 126\ 84\ 36\ 9\ 1$.

\section*{13}
\begin{enumerate}[a]
	\item %a
	In the first case you choose $r$ elements out of a set of $n$ elements, next you choose $k$ elements from the set you chose in step one.\\
	In the end you are left with a pile of $k$ elements that were chosen in step 2, and a pile of $r - k$ elements, that were not chosen in step 2.

	The second method is a bit more direct. You immediately choose $k$ elements from a set of $n$ elements. This is the same as the first pile from method one. Now you are left with $n - k$ elements, of which you choose $r - k$ to match the pile that you chose in step 1 of method 1, but that wasn't chosen in step 2.\\
	In the end you are again left with two piles of identical sizes as in method 1.

	Both methods of choosing elements are equivalent.

	\item %b
	$\begin{pmatrix}
		n \\ r
	\end{pmatrix}
	* \begin{pmatrix}
		r \\ k
	\end{pmatrix}\\
	= \dfrac{n!}{r! (n - r)!} * \dfrac{r!}{k! (r - k)!}\\
	= \dfrac{n! r!}{r! (n - r)! k! (r - k)!}\\
	= \dfrac{n!}{(n - r)! k! (r - k)!}\\
	= \dfrac{n!}{k! (r - k)! (n - r)!}\\
	= \dfrac{n! (n - k)!}{k! (n - k)! (r - k)! (n - r)!}\\
	= \dfrac{n!}{k! (n - k)!} * \dfrac{(n - k)!}{(r - k)! (n - r)!}\\
	= \dfrac{n!}{k! (n - k)!} * \dfrac{(n - k)!}{(r - k)! (n - k - r + k)!}\\
	= \begin{pmatrix}
		n \\ k
	\end{pmatrix}
	* \begin{pmatrix}
		n - k \\ r - k
	\end{pmatrix}$
\end{enumerate}

\section*{14}
$(x + y)^n = \sum_{k = 0}^{n} C(n, k) x^{n - k} y^k$

For $x = 1$ we can simplify this to $(1 + y)^n = \sum_{k = 0}^{n} C(n, k) y^k$.

$\sum_{k = 0}^{n} C(n, k) \sqrt{2}^k + \sum_{k = 0}^{n} C(n, k) (-\sqrt{2})^k\\
=\sum_{k = 0}^{n} C(n, k)(\sqrt{2}^k + (-\sqrt{2})^k)$


For all even numbers $\sqrt{2}^k, (-\sqrt{2})^k \in \mathbb{N}$ holds. It also holds that $\sum_{n \in \mathbb{N}} n \in \mathbb{N}$.\\
For uneven numbers $\sqrt{2}^k + (-\sqrt{2})^k = 0$ holds.

We can conclude that $\forall n \in \mathbb{N}, (1 + \sqrt{2})^n + (1 - \sqrt{2})^n \in \mathbb{N}$.

\section*{15}
\begin{enumerate}[a]
	\item %a
	\begin{enumerate}[i]
		\item %i
		$\begin{pmatrix}
			123456 \\ 123457
		\end{pmatrix} = 0$, because $123457 > 123456$.
		\item %ii
		$\begin{pmatrix}
			3 \pi \\ 2 \pi
		\end{pmatrix}$ is not defined, because $\begin{pmatrix}
			u \\ k
		\end{pmatrix}$ is only defined for $u \in \mathbb{R}, k \in \mathbb{Z}^+$, and $2 \pi \not \in \mathbb{Z}^+$.
		\item %iii
		According to the extended binomial coefficient definition $\begin{pmatrix}
			u \\ k
		\end{pmatrix} = \dfrac{u (u - 1) \dots (u - k + 1)}{k!}$.

		$\begin{pmatrix}
			-\sqrt{5} \\ 4
		\end{pmatrix}\\
		= \dfrac{-\sqrt{5} (-\sqrt{5} - 1) (-\sqrt{5} - 2) (-\sqrt{5} - 3)}{4!}\\
		= \dfrac{-\sqrt{5} (-\sqrt{5} - 1) (-\sqrt{5} - 2) (-\sqrt{5} - 3)}{24}\\
		= \dfrac{(5 + \sqrt{5}) (-\sqrt{5} - 2) (-\sqrt{5} - 3)}{24}\\
		= \dfrac{(-5 \sqrt{5} - 10 - 5 - 2 \sqrt{5}) (-\sqrt{5} - 3)}{24}\\
		= \dfrac{(-7 \sqrt{5} - 15) (-\sqrt{5} - 3)}{24}\\
		= \dfrac{35 + 21 \sqrt{5} + 15 \sqrt{5} + 45}{24}\\
		= \dfrac{36 \sqrt{5} + 80}{24}\\
		= \dfrac{9 \sqrt{5} + 20}{6}\\
		= \dfrac{10}{3} + \dfrac{3}{2} \sqrt{5}$
	\end{enumerate}
	\item %b
	If you calculate $\sum_{i = 0}^{n} \begin{pmatrix}
		n \\ i
	\end{pmatrix}$ you are calculating and summing the sizes of all possible subsets. We know that the power set $\mathcal{P}$ is the set of all subsets. We also know that the size of $\mathcal{P}$ is $2^n$ because for every element in the set, there are 2 possibilities: either it is included or it isn't.

	Thus we can conclude that $\sum_{i = 0}^{n} \begin{pmatrix}
		n \\ i
	\end{pmatrix} = | \mathcal{P} | = 2^n$.
	\item %c
	$(x + y)^n = \sum_{k = 0}^n \begin{pmatrix}
		n \\ k
	\end{pmatrix} x^{n - k} y^k$

	In the case of $(3x + 4y)^{13}$ we have $n = 13$, and in the term $x^5 y^8$ we have $k = 8$, so the coefficient is $\begin{pmatrix}
		13 \\ 8
	\end{pmatrix} = 1287$.
\end{enumerate}

\end{document}
